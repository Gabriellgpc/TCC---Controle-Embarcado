% resumo na língua vernácula (obrigatório)
\setlength{\absparsep}{18pt} % ajusta o espaçamento dos parágrafos do resumo
\begin{resumo}

Este trabalho apresenta uma proposta e testa na prática um sistema de observação e controle de velocidade para motores de corrente contínua equipados com \emph{encoders} de baixa resolução. O trabalho tem como aplicação alvo mini robôs de acionamento diferencial. O sistema proposto foi implementado no microcontrolador (\emph{ESP32}) e mostrou, pelos resultados experimentais, ser capaz de controlar a velocidade do par motor-roda de robôs com acionamento diferencial (com \emph{encoders} rotativos de baixa precisão) de forma a conseguir reduzir as assimetrias dos motores. Foi feito o uso de um esquema de controle com duas estratégias, do tipo \emph{Feedforward}/\emph{Backward} e usou-se o filtro de \emph{Kalman} para realizar a estimativa da velocidade, com o objetivo de reduzir o erro de quantização dos sensores.

% Escreva seu resumo aqui. Ele deve ser parágrafo único e sem récuo na primeira linha.
%  \lipsum[1-1]
 
%  \noindent
 \textbf{Palavras-chaves}: Controle Embarcado. Motores CC. Filtro de \emph{Kalman}. \emph{ESP32}. Robôs com acionamento diferencial.
\end{resumo}
% ---
% resumo em inglês
\begin{resumo}[Abstract]
	\begin{otherlanguage*}{english}
	
		This work presents a proposal and tests in practice a speed observer and controller for DC motors equipped with low-resolution encoders. The target application of this work are mini robots with differential drive. The system was implemented in the microcontroller (ESP32). It demonstrated, by experimental results, to be able to control the speed of the motor-wheel pairs of differential drive robots (with low-precision rotary encoders) to reduce their asymmetries. A control scheme with two control strategies, of the type Feedforward / Backward, was used, with the Kalman filter as the speed estimator to reduce the sensors quantization error.
	
	\vspace{\onelineskip}
	\noindent 
	\textbf{Keywords}: Embedded Control. DC Motors. Kalman Filter. ESP32. Robots with differential Drive.
	\end{otherlanguage*}
\end{resumo}