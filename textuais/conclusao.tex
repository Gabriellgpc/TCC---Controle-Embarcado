\chapter[Conclusão]{Conclusão}
\label{ch:conclusao}
% TER EM MENTE:
% Deve ser fundamentada nos resultados, contendo deduções lógicas que correspondam aos objetivos do tema proposto e às expectativas descritas pelo autor na introdução do trabalho. É a resposta à pergunta da pesquisa – o objetivo do TCC. A conclusão é a fase final de toda a argumentação; relaciona as diversas partes da argumentação, amarra as ideias desenvolvidas. É a síntese de toda reflexão e, em certo sentido, é um regresso à introdução: fecha-se o começo, o que se propôs na introdução.

% Escreva suas conclusões, limitações do seu trabalho, contribuições, trabalhos futuros, etc.

Pelos resultados apresentados no \autoref{ch:resultados}

Alguns conclusões importantes podem ser feita com relação ao exporto no \autoref{ch:resultados}, são elas: A aproximação do comportamento dos motores para um sistema de primeira ordem, mostrou-se ser suficientemente satisfatória para todos os experimentos realizados (como esperado era de se esperar, conforme apresentado na \autoref{sec:motor_ref_teo}); O uso do filtro de \emph{Kalman} como estimador ótimo mostrou-se ser extremamente eficiente para contornar o erro de quantização (como esperado, conforme \cite{enco}) dos sensores de baixa resolução
