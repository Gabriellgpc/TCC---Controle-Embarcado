\chapter[Conclusão]{Conclusão}
\label{ch:conclusao}
% TER EM MENTE:
% Deve ser fundamentada nos resultados, contendo deduções lógicas que correspondam aos objetivos do tema proposto e às expectativas descritas pelo autor na introdução do trabalho. É a resposta à pergunta da pesquisa – o objetivo do TCC. A conclusão é a fase final de toda a argumentação; relaciona as diversas partes da argumentação, amarra as ideias desenvolvidas. É a síntese de toda reflexão e, em certo sentido, é um regresso à introdução: fecha-se o começo, o que se propôs na introdução.

% Escreva suas conclusões, limitações do seu trabalho, contribuições, trabalhos futuros, etc.

Pelo exposto no \autoref{ch:resultados} podemos notar que: 

\begin{itemize}
    \item A aproximação do comportamento dos motores para um sistema de primeira ordem, mostrou-se ser suficientemente para todos os experimentos realizados (como esperado, conforme apresentado na \autoref{sec:motor_ref_teo});
    \item O uso do filtro de \emph{Kalman} para estimar a velocidade de rotação dos motores, mostrou-se eficiente, pois conseguiu reduzir quase que completamente os erros de quantização, mesmo com medições provenientes de sensores de baixa resolução e operando em altas velocidades (pior condição para a leitura de \emph{Encoders} pelo método da medição de períodos);
    \item O controlador \emph{Feedforward} em conjunto com o controlador proporcional (\emph{Backward}) também mostrou bons resultados, conseguindo rastrear a referência com uma margem de erro relativamente baixa, além de atingir os tempos de respostas desejados (constante de tempo desejada) e contornar as pertubações externas (como mostrado principalmente no experimento 4 nas Figuras \ref{fig:exp04:antes_vs_depois}).\\
\end{itemize}


Conclui-se portanto que o sistema proposto é capaz de oferecer boas estimativas para as velocidades dos motores, por meio do filtro de \emph{Kalman}, mesmo para sensores com baixa resolução. E que o sistema de controle (fazendo uso dessas estimativas) é capaz de reduzir as assimetrias do par motor-roda de (mas não limitando-se) robôs com acionamento diferencial. Além disso, o sistema proposto apresentou-se como uma solução simples o suficiente para conseguir ser implementado em um \emph{Hardware} com recursos limitados, como um  microcontrolador.\\

Fazem-se necessárias também algumas observações importantes a respeito dos experimentos e de suas análises apresentados no capítulo anterior. São elas: 

\begin{itemize}
    \item A escolha de se trabalhar com as velocidades no eixo do motor antes da caixa de redução fez com que os valores em módulo dessas velocidades fossem altos, dificuldade um pouco a análise visual dos gráficos, fazendo (por exemplo) com que erros absolutos no rastreio da referência sejam mascarados devido à escala dos gráficos.
    
    \item Outro ponto é com relação à variedade dos experimentos: os quatro experimentos mostram pouca diferença entre si. Provavelmente isso é devido ao fato de que a maioria deles foram realizados com os robôs suspensos (sem contato com o chão), apenas o experimento 4 foi feito com um robô em contato com o chão (situação de operação normal do robô). O motivo disso foram problemas técnicos encontrados durante a fase de experimentação e testes, que fez com que dois dos três robôs disponíveis não conseguissem operar normalmente, limitando assim as possibilidades de testes.\\
\end{itemize}


O presente trabalho possui como objetivos futuros: 

\begin{itemize}
    \item A adição da ação integral (controlador PI) no sistema de controle, para eliminar o erro residual que o controlador \emph{Feedforward} não está sendo capaz de eliminar;
    
    \item E a calibração automática (dinâmica), principalmente do ganho do sistema, por meio da técnica de mínimos quadrados recursivos, como apresentado em alguns dos trabalhos descritos na \autoref{sec:trabalhos_relacionados}.
\end{itemize}

 

