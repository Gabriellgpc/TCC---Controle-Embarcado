\section{Mínimos Quadrados}
\label{sec:MMQ}
De maneira geral, a ideia por trás do método dos mínimos quadrados (MMQ)  é de encontrar os coeficientes da(s) funções de base que minimizem o erro, ou seja, com o menor erro possível associado a representação \cite{MMQ}. Assim, dado que queremos obter a equação e possuímos vetores de dados $x$ e $y$, e queremos relacioná-los. Por exemplo, caso queira-se aproximar o conjunto de dados para a reta:

\begin{equation}
    y = ax + b.
    \label{eq:pol1}
\end{equation}

Teremos a seguinte relação para o MMQ.

\begin{align*}
    \textbf{Y} = \textbf{X}\alpha
\end{align*}
% \[\textbf{Y} = \textbf{X}\alpha\]
sendo,

\begin{equation*}
    \textbf{Y} =
    \begin{bmatrix}
     y_1\\
     y_2\\
     \vdots\\
     y_n
    \end{bmatrix},
    \textbf{X} = 
    \begin{bmatrix}
        x_1 & 1\\
        x_2 & 1\\
        \vdots & \vdots\\
        x_n & 1\\
    \end{bmatrix},
    \alpha =
    \begin{bmatrix}
        a\\
        b\\
    \end{bmatrix}
\end{equation*}

A ideia por trás do método de mínimos quadrados é minimizar o erro $e$. Sabendo que o espaço gerado por tal plano ortogonal será dado pela matriz $X^T$ , então queremos impor que o vetor
de erro $e$ será nulo, logo:

\[\textbf{X}^Te = 0\]

Como,

\[e = \textbf{Y} - \textbf{X}\hat{\alpha}\]

Então temos,

\[\textbf{X}^T(\textbf{Y} - \textbf{X}\hat{\alpha})= 0\]
\[\hat{\alpha} = (\textbf{X}^T\textbf{X})^{-1}\textbf{X}^T\textbf{Y} \]

Sendo,

\[\hat{\alpha} = 
\begin{bmatrix}
 \hat{a}\\
 \hat{b}\\
\end{bmatrix}
\]

Os elementos do vetor $\hat{\alpha}$, Serão os coeficientes da equação \ref{eq:pol1} que minimizarão o erro, ou seja, são os coeficientes da reta que melhor representa o conjunto de dados.