\section{Filtro de Kalman}
% Introdução ao filtro de Kalman
\label{sec:kalman}
"A filtragem de \emph{Kalman} é um processo de estimativa de estado ótimo aplicado a um sistema dinâmico que envolve perturbações aleatórias. Mais precisamente, o filtro de \emph{Kalman} fornece um  procedimento recursivo linear, imparcial e que visa minimizar a variância do erro para estimar de forma otimizada o estado desconhecido de um sistema dinâmico de dados ruidosos obtidos em tempo real discreto. Tem sido amplamente utilizado em muitas áreas de aplicações industriais e governamentais, como sistemas de rastreamento, navegação por satélite, estimar trajetória de mísseis balísticos e radar." \cite{kalman:book}.\\

O filtro de \emph{Kalman} usa um modelo dinâmico do sistema (por exemplo, leis físicas do movimento), entradas de controle conhecidas para esse sistema e várias medições sequenciais (como de sensores) para formar uma estimativa das quantidades variáveis do sistema (seu estado $\textbf{x}_k$) que é melhor do que a estimativa obtida usando apenas as medições. Como tal, é um algoritmo comum de fusão de sensores e fusão de dados \cite{kalman:wiki}, \cite{Kalman:ofid}.\\

O filtro produz uma estimativa do estado do sistema como uma média do estado previsto do sistema e da nova medição usando uma média ponderada. O objetivo dos pesos é que os valores com menor incerteza sejam mais "confiáveis". O ganho de \emph{Kalman} é o peso relativo dado às medições e estimativa do estado atual.\\

Para o uso do filtro de \emph{Kalman} deve-se modelar o processo de acordo com a seguinte estrutura. Isso significa especificar as seguintes matrizes \cite{kalman:book}:

\begin{itemize}
    \item $\textbf{F}_k$: Modelo de transição de estado.
    \item $\textbf{H}_k$: Modelo de Observação.
    \item $\textbf{Q}_k$: Covariância do ruído do processo.
    \item $\textbf{R}_k$: Covariância do ruído da observação(medição).
    \item $\textbf{B}_k$: Modelo de entrada de controle (caso o sistema dinâmico tenha entrada). 
\end{itemize}


Modelo do sistema para o filtro de \emph{Kalman}:
\begin{equation}
\textbf{x}_k = \textbf{F}_x \textbf{x}_{k-1} + \textbf{B}_k \textbf{u}_k + \textbf{w}_k; \textbf{w}_k \sim N(0, \textbf{Q}_k)
\end{equation}

Sendo $\textbf{w}_k$ o ruído de processo, que assume-se ter uma distribuição normal de média zero e covariância $\textbf{Q}_k$. ($\textbf{w}_k \sim N(\textbf{0}, \textbf{Q}_k)$).

Modelo da observação/medição será:
\begin{equation}
\textbf{z}_k = \textbf{H}_k \textbf{x}_k + \textbf{v}_k; \textbf{v}_k \sim N(\textbf{0}, \textbf{R}_k)
\end{equation}

Sendo $\textbf{v}_k$ o ruído de observação, que assume-se ser um ruído gaussiano branco de média zero e covariância $\textbf{R}_k$. ($\textbf{v}_k \sim N(\textbf{0}, \textbf{R}_k)$).

A filtragem é frequentemente dividida como duas fases: Predição e Atualizar (Alguns autores também contam a etapa de medição como uma fase do algoritmo). A fase de predição usa a estimativa anterior para produzir uma estimativa do estado atual. Na fase seguinte (de atualização), a predição atual é combinada com as informações da medição atual para aprimorar a estimativa do estado. As etapas do filtro são apresentadas a seguir.

\textbf{Predição}:
\begin{align*}
    \check{\textbf{x}}_k &= \textbf{F}_k \hat{\textbf{x}}_{k-1} + \textbf{B}_k \textbf{u}_k\\
    \check{\textbf{P}}_k &= \textbf{F}_k \hat{\textbf{P}}_{k-1} \textbf{F}^T_k + \textbf{Q}_k
\end{align*}

Sendo $\check{\textbf{P}}$ uma predição da covariância do estado seguinte (predito, $\textbf{x}_k$).

\textbf{Atualização}:
\begin{align*}
    \textbf{K}_k &= \check{\textbf{P}}_k \textbf{H}^T \left( \textbf{H}_k \check{\textbf{P}}_k \textbf{H}^T_k + \textbf{R}_k\right)^{-1}\\
    \hat{\textbf{x}}_k &= \check{\textbf{x}}_k + \textbf{K}_k\left( \textbf{z}_k - \textbf{H}_k \check{\textbf{x}}_k \right)\\
    \hat{\textbf{P}}_k &= \left(\textbf{I} - \textbf{KH}_k \right)\check{\textbf{P}}_k
\end{align*}

Uma observação importante sobre a etapa de atualização é que ela irá tender para o valor predito caso o ganho do filtro $\textbf{K}_k$ seja baixo (resultado de um alto erro na medição). O contrário também ocorre, ou seja, para medições precisas ($\textbf{R}_k$ baixo) o ganho do filtro tende a ser alto e a etapa de atualização vai tender a usar mais a informação da medição como estado ótimo ($\hat{\textbf{x}}$). 
