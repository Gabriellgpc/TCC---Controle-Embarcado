\section{Modulação por Largura de Pulso}
\label{sec:PWM}
% Resumo sobre PWM

\emph{Pulse Width Modulation} (PWM) refere-se a um sinal digital pulsante. Esse sinal é utilizado, por exemplo, para simular uma saída analógica em microcontroladores e por isso geralmente é muito usado para o controle de atuadores elétricos como motores, aquecedores, dentre outras coisas, como o controle do brilho de LEDs. \\

O \emph{PWM} pode ser visto como uma maneira de codificar digitalmente níveis de sinal analógico. Nesta técnica, através do uso de contadores de alta resolução, o ciclo de trabalho de uma onda quadrada é modulado para codificar um nível de sinal analógico específico para que então ele atenda os requisitos de uma aplicação desejada. O tempo de ativação é o tempo durante o qual a alimentação CC é aplicada (sinal em nível alto) à carga e o tempo de desativação é o período durante o qual a alimentação é desligada (sinal em nível baixo). Dada uma largura de banda suficiente, qualquer valor analógico pode ser codificado com \emph{PWM}.